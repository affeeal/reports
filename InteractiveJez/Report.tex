\documentclass{article}

\usepackage[T1,T2A]{fontenc}
\usepackage[utf8]{inputenc}
\usepackage[english,russian]{babel}
\usepackage{fancyvrb}

\begin{document}

\section{Основные структуры данных}

Уравнение в свободных полгруппах \texttt{t.Eq} представляется в виде
\begin{Verbatim}
  t.Eq ::= ((AreEqual (e.LHS) (e.RHS)) (e.Constrs) (e.Conds)).
\end{Verbatim}
Выражения \texttt{e.LHS} и \texttt{e.RHS} - сооответственно левая и правая
части уравнения, состоящие из произвольного числа констант \texttt{t.Const} и
переменных \texttt{t.Var}.
\begin{Verbatim}
  t.Const ::= (s.CHAR s.NUMBER);
  t.Var ::= (Var s.CHAR).
\end{Verbatim}
Выражение \texttt{e.Constrs} представляет \textit{ограничения на переменные}
уравнения в конъюнктивной нормальной форме. Элементами дизъюнкций являются
\textit{рестрикции} \texttt{t.Restr}. Рестрикции бывают \textit{суффиксными}
\texttt{t.SuffixRestr}, \textit{префиксными} \texttt{t.PrefixRestr} и
рестрикциями \textit{на пустоту} \texttt{t.EmptyRestr}:
\begin{Verbatim}
  t.Restr ::= t.SuffixRestr | t.PrefixRestr | t.EmptyRestr;
  t.SuffixRestr ::= (not t.Const ends t.Var);
  t.PrefixRestr ::= (not t.Const starts t.Var);
  t.EmptyRestr ::= (not empty t.Var).
\end{Verbatim}
Ограничение может включать в себя одну или две рестрикции и называться в
соответствии с этим \textit{тривиальным} \texttt{t.TrivialConstr} или
\textit{нетривиальным} \texttt{t.NonTrivialConstr}. Таким образом, существует
четыре вида ограничений.
\begin{Verbatim}
  t.Constr ::= t.TrivialConstr | NonTrivialConstr;
  t.TrivialConstr ::= (OR t.Restr);
  t.NonTrivialConstr ::= (OR t.SuffixRestr t.PrefixRestr).
\end{Verbatim}
Наконец, выражение \texttt{e.Conds} представляет \textit{условия на константы}.
Константа может являться результатом сжатия в пару или в блок, поэтому
выделяют два вида условий:
\begin{Verbatim}
  t.Cond ::= t.PairCond | t.BlockCond;
  t.PairCond ::= (t.Const is t.SimpleBlock t.SimpleBlock);
  t.BlockCond ::= (t.Const is t.ArbitaryBlock),
\end{Verbatim}
где блок \texttt{t.Block} есть степень константы с показателями
\texttt{t.Exp}:
\begin{Verbatim}
  t.Block ::= t.SimpleBlock | t.ArbitaryBlock;
  t.SimpleBlock ::= (t.Const /* нет иных показателей */ (const 1));
  t.ArbitaryBlock ::= (t.Const t.Exp* (const s.NUMBER));
  t.Exp ::= (s.WORD s.NUMBER).
\end{Verbatim}

\end{document}
